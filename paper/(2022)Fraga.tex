%% 
%% Copyright 2007-2020 Elsevier Ltd
%% 
%% This file is part of the 'Elsarticle Bundle'.
%% ---------------------------------------------
%% 
%% It may be distributed under the conditions of the LaTeX Project Public
%% License, either version 1.2 of this license or (at your option) any
%% later version.  The latest version of this license is in
%%    http://www.latex-project.org/lppl.txt
%% and version 1.2 or later is part of all distributions of LaTeX
%% version 1999/12/01 or later.
%% 
%% The list of all files belonging to the 'Elsarticle Bundle' is
%% given in the file `manifest.txt'.
%% 
%% Template article for Elsevier's document class `elsarticle'
%% with harvard style bibliographic references

\documentclass[preprint,12pt,authoryear]{elsarticle}

%% Use the option review to obtain double line spacing
%% \documentclass[authoryear,preprint,review,12pt]{elsarticle}

%% Use the options 1p,twocolumn; 3p; 3p,twocolumn; 5p; or 5p,twocolumn
%% for a journal layout:
%% \documentclass[final,1p,times,authoryear]{elsarticle}
%% \documentclass[final,1p,times,twocolumn,authoryear]{elsarticle}
%% \documentclass[final,3p,times,authoryear]{elsarticle}
%% \documentclass[final,3p,times,twocolumn,authoryear]{elsarticle}
%% \documentclass[final,5p,times,authoryear]{elsarticle}
%% \documentclass[final,5p,times,twocolumn,authoryear]{elsarticle}

%% For including figures, graphicx.sty has been loaded in
%% elsarticle.cls. If you prefer to use the old commands
%% please give \usepackage{epsfig}

%% The amssymb package provides various useful mathematical symbols
\usepackage{amssymb}
%% The amsthm package provides extended theorem environments
%% \usepackage{amsthm}

%% The lineno packages adds line numbers. Start line numbering with
%% \begin{linenumbers}, end it with \end{linenumbers}. Or switch it on
%% for the whole article with \linenumbers.
%% \usepackage{lineno}

\journal{Br. J. of Proc. Anal., Mod. and Opt.}

\begin{document}

\begin{frontmatter}

%% Title, authors and addresses

%% use the tnoteref command within \title for footnotes;
%% use the tnotetext command for theassociated footnote;
%% use the fnref command within \author or \affiliation for footnotes;
%% use the fntext command for theassociated footnote;
%% use the corref command within \author for corresponding author footnotes;
%% use the cortext command for theassociated footnote;
%% use the ead command for the email address,
%% and the form \ead[url] for the home page:
%% \title{Title\tnoteref{label1}}
%% \tnotetext[label1]{}
%% \author{Name\corref{cor1}\fnref{label2}}
%% \ead{email address}
%% \ead[url]{home page}
%% \fntext[label2]{}
%% \cortext[cor1]{}
%% \affiliation{organization={},
%%            addressline={}, 
%%            city={},
%%            postcode={}, 
%%            state={},
%%            country={}}
%% \fntext[label3]{}

\title{Practice shows the way}

%% use optional labels to link authors explicitly to addresses:
%% \author[label1,label2]{}
%% \affiliation[label1]{organization={},
%%             addressline={},
%%             city={},
%%             postcode={},
%%             state={},
%%             country={}}
%%
%% \affiliation[label2]{organization={},
%%             addressline={},
%%             city={},
%%             postcode={},
%%             state={},
%%             country={}}

\author{Tatiana Balbi Fraga}

\affiliation{organization={Núcleo de Tecnologia, Universidade Federal de Pernambuco},%Department and Organization
            addressline={}, 
            city={},
            postcode={}, 
            state={},
            country={}}

\begin{abstract}
This paper presents a university teaching strategy, grounded on a practice-based learning approach. 
 
\end{abstract}

%%Graphical abstract
\begin{graphicalabstract}
\includegraphics[width=\textwidth]{figures/graphicalAbstract.png}
\end{graphicalabstract}

%%Research highlights
\begin{highlights}
\item This paper presents a strategy for teaching undergraduate students based on practical experience
\item This strategy is strongly associated with the inseparability between teaching, research, extension and technological development
\item This paper also recommends some important resources that can be used to develop group work
\item And it discusses a solution to the impasse between transparency and guaranteeing the authenticity of the work, already widely used for software development but not in the development of research works
\end{highlights}

\begin{keyword}
%% keywords here, in the form: keyword \sep keyword
practical-based learning

%% PACS codes here, in the form: \PACS code \sep code

%% MSC codes here, in the form: \MSC code \sep code
%% or \MSC[2008] code \sep code (2000 is the default)

\end{keyword}

\end{frontmatter}

%% \linenumbers

%% main text

\section{Introduction}
\label{}

Problem-based learning (PBL) is a pedagogical approach applied worldwide, especially in nursing and medicine courses. 

\section{Practice-based learning}
\label{}

Practice-based learning strategy starts with students learning the basics of some science in the classroom. Concomitantly or after this first learning process, students carry out visits to an organization, through which they will be able to understand organizational processes, identifying aspects of the process that need to be better managed. Students then prepare a report of the visit, describing in detail the process studied and the real problems identified. Subsequently, this information is taken to the Teacher-Facilitator. Through discussions between the teacher and the students involved, the real problems, that is, the aspects of the studied process that need to be improved, are formatted as decision problems or optimization problems.

For a better understanding of the difference between the real problem identified by the students and the formatted problem, consider the following example: after a student has visited a company that performs the maintenance service of hospital infusion pumps and analyzed in detail the processes of this company, she identified the following real problem:

\begin{quote}
"The company is unable to correctly program the hospital pump overhaul service. There are delays in maintenance and services are not prioritized correctly."
\end{quote}

After an in-depth study of the company's processes, and extensive discussions between teacher and student, it was identified that the problem was not associated with the maintenance process, but with the scheduling of equipment collection in hospitals, maintenance and return. In this context, a new allocation and routing problem was identified, defined as 'Problem of Planning the Maintenance and Transport of Hospital Infusion Pumps'.

After identifying the decision/optimization problem...

E então é feito um levantamento bibliográfico sobre problemas próximos aos problemas de decisão / otimização identificados e sobre metodologias que são aplicadas na solução destes problemas. O professor pesquisador modela tais problemas e desenvolve algoritmos de solução para os modelos desenvolvidos. Todo o conhecimento em desenvolvimento é então passado para os alunos que, por sua vez, serão motivados a participar do processo de modelagem e solução dos problemas identificados. 

Mathematical Modeling and Optimization are extremely relevant sciences for students in production engineering. Especially when we experience the expansion of Industry 4.0 (Ram, 2021). Through Mathematical Modeling it is possible to accurately understand any industrial or organizational problems in which process optimization is sought and, once a model is built, in order to find the desired answer to the problem, or at least a somewhat close answer, Optimization, and so Programming, simply becomes indispensable. Through my experiences at the university where I work, I developed the following strategy for joint teaching of
Mathematical Modeling, Programming and Optimization, combining teaching, extension, research and technology:

Teaching: the fundamentals of mathematical modeling and optimization can be taught to
students in two different subjects: 1) Mathematical programming; and 2) Mathematical
Modeling. In the first course, modeling mathematical programming problems is taught along
with mathematical methods of solution. Modeling problems close to real problems can and
should be motivated through case modeling exercises such as those proposed by Hillier and
Lieberman (2010). In the second course, students should learn how to model standard
optimization problems (e.g. allocation problems; balancing problems; routing problems; and
scheduling problems) in addition to some heuristics used to solve such problems. In both
courses, students must be motivated to understand and present scientific papers with the
modeling of real problems.
Extension: the extension applies in four moments: 1) students look for a company to develop
their work, know the company, understand the company's operating process, and help to
identify optimization problems within the company; 2) an Applied Programming course is
offered to students, through which students learn object-oriented programming; 3) the
students work with the company in the search for ideas and the refinement of the proposed
solution; 4) students take the solution developed for different companies, conducting data
collections and testing the solver.
Research: once a real problem is identified; it is necessary to carry out a vast bibliographic
survey, identifying mathematical models and proposed solution approaches for problems close
to the one studied, and it is necessary to develop new solution algorithms, through the
adaptation of solutions proposed in the literature. The research must be carried out through
research projects, and possibly in parallel with the development of course conclusion works
and scientific and technological initiation projects.
Technological: once the solution is developed and tested, it is recommended to develop a
solver, with a good graphical interface, that can be easily used by company employees. This
step can be carried out through technological projects, developed in parallel with graduation
projects for students in the technological area, and projects of technological initiation. At this
stage, it is also possible to develop partnerships with startup companies, and acquire financial
resources to support research and innovation developed within the university.
The four lines of action are related as follows: through teaching, students acquire the basic
knowledge necessary for the development of research and extension activities; through the
extension, students deepen their knowledge of programming, help identify new optimization
problem and approaches applied by the company to solve these problems and test the
solutions developed, also qualifying company employees; through research, students
participate in the development of solutions to identified problems; and through technological
projects, students can participate in software development. Both research and extension
works contribute to teaching, as they allow students to understand and expand all the content
presented in the classroom, through practice. It is important to emphasize that throughout the
process, students must work together with teachers who have extensive knowledge and
practice regarding the subjects covered. So, the teachers involved also deepen their
knowledge, which significantly improves the quality of teaching.

%% The Appendices part is started with the command \appendix;
%% appendix sections are then done as normal sections
%% \appendix

%% \section{}
%% \label{}

%% If you have bibdatabase file and want bibtex to generate the
%% bibitems, please use
%%
%%  \bibliographystyle{elsarticle-harv} 
%%  \bibliography{<your bibdatabase>}

%% else use the following coding to input the bibitems directly in the
%% TeX file.

\begin{thebibliography}{00}

%% \bibitem[Author(year)]{label}
%% Text of bibliographic item

\bibitem[ ()]{}

\end{thebibliography}
Donald E. Knuth (1986) \emph{The \TeX{} Book}, Addison-Wesley Professional.
\end{document}

\endinput
%%
%% End of file `elsarticle-template-harv.tex'.

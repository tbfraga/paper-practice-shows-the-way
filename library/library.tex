\documentclass{article}
\usepackage[utf8]{inputenc}

% Used in the explanation text
\usepackage{hyperref}
\hypersetup{
    colorlinks = true,
    citecolor = {blue},
    urlcolor = {blue},
}

% Used by the template
\usepackage{setspace}
\usepackage{changepage} % to adjust margins
\usepackage[breakable]{tcolorbox}
\usepackage{float} % for tables inside tcolorbox https://tex.stackexchange.com/a/274342
\usepackage{enumitem}
\usepackage[authoryear, round]{natbib}


\newtcolorbox[auto counter]{pabox}[2][]{%
colback=white!10!white, breakable, sharp corners, boxrule=0.7pt, title= \textbf{Problem Based Learning ~\thetcbcounter #2}, #1}

\title{Library record}
\author{Tatiana Balbi Fraga}
\date{Sept 2022}

\begin{document}

\maketitle

% Each section is supposed to be brief, in the form of a bullet list.
% This environment formats the lists in each model card section in a compact format to help
% the card fit into the recommended "one to two pages".
\newenvironment{mcsection}[1]
    {%
        \textbf{#1}

        % Reduce margins to use the space more effectively and help fit in the recommended "one to two pages"
        % Use the bullet list format as shown in the model card paper to increase readability
        \begin{itemize}[leftmargin=*,topsep=0pt,itemsep=-1ex,partopsep=1ex,parsep=1ex,after=\vspace{\medskipamount}]
    }
    {%
        \end{itemize}
    }

% Optional: reduce margins single line to fit in "one to two pages", as recommended
\begin{adjustwidth}{-60pt}{-30pt}
\begin{singlespace}

\tcbset{colback=white!10!white}

\begin{minipage}{1.26\textwidth}

\begin{pabox}[label={myusecounter}]{\hfill  \cite{Fernandes2021}}
% Change to a smaller, but still legible font size to help fit in the recommended "one to two pages"

\textbf{brief:} \\

The author portrays his experience related to PBL approach, both as a student and as a tutor. He reports how important it is for the tutor to be prepared to apply this approach and reports that an unprepared tutor can generate a strongly negative experience for the students, harming the learning process.\\

\textbf{PBL definition:}
\begin{quote}
 ``Problem-based learning (PBL) is an active learning method that allows for students to self-identify their learning needs and work together in small groups to achieve their learning objectives."
\end{quote} 

\textbf{PBL method} \\

It is applied as a discipline at the end of the pharmacy course which lasts for 8 weeks. Groups of eight students are formed, and a different tutor, with experience in the area, is assigned to each group. A problem is presented to the students, and the group must work on solving this problem by exchanging experiences and learning. Each student identifies gaps in his / her own knowledge, and has the mission to delve deeper into the issues related to the identified gaps. The groups have weekly meetings where they must discuss the subject and define new learning goals, always focusing on solving the problem. \\

\textbf{useful:} \\

The author presents a good example of application of Borton’s Development Framework.

\tcblower

\bibliographystyle{unsrtnat}
\bibliography{bibFile}

\end{pabox}
\end{minipage}

\tcbset{colback=white!10!white}
\begin{tcolorbox}[title=\textbf{Model Card - CheXNet},
    breakable, sharp corners, boxrule=0.7pt]

% Change to a smaller, but still legible font size to help fit in the recommended "one to two pages"
\small{

Refer to section 4.1 of the \href{https://arxiv.org/abs/1810.03993}{model card paper } -- remove this line after filling in this section.

\begin{mcsection}{Model Details}
    \item Detail 1...
    \item Detail 2...
\end{mcsection}

Refer to section 4.2 of the \href{https://arxiv.org/abs/1810.03993}{model card paper } -- remove this line after filling in this section.

\begin{mcsection}{Intended Use}
    \item Intended use 1...
\end{mcsection}

Refer to section 4.3 of the \href{https://arxiv.org/abs/1810.03993}{model card paper } -- remove this line after filling in this section.

\begin{mcsection}{Factors}
    \item Factors 1...
\end{mcsection}

Refer to section 4.4 of the \href{https://arxiv.org/abs/1810.03993}{model card paper } -- remove this line after filling in this section.

\begin{mcsection}{Metrics}
    \item Metrics 1....
\end{mcsection}

Refer to section 4.5 of the \href{https://arxiv.org/abs/1810.03993}{model card paper } -- remove this line after filling in this section.

\begin{mcsection}{Evaluation Data}
    \item Evaluation data 1...
\end{mcsection}

Refer to section 4.6 of the \href{https://arxiv.org/abs/1810.03993}{model card paper } -- remove this line after filling in this section.

\begin{mcsection}{Training Data}
    \item Training data 1...
\end{mcsection}

\pagebreak

Refer to section 4.8 of the \href{https://arxiv.org/abs/1810.03993}{model card paper } -- remove this line after filling in this section.

\begin{mcsection}{Ethical Considerations}
    \item Ethical considerations 1....
\end{mcsection}

Refer to section 4.9 of the \href{https://arxiv.org/abs/1810.03993}{model card paper } -- remove this line after filling in this section.

\begin{mcsection}{Caveats and Recommendations}
    \item Caveats and recommendations 1...
\end{mcsection}

Refer to section 4.7 of the \href{https://arxiv.org/abs/1810.03993}{model card paper } -- remove this line after filling in this section.

\textbf{Quantitative Analyses}

% Sample table inside tcolorbox
\begin{table}[H]
\centering
\small{
\begin{tabular}{lr}
Measurement 1       & 0.751  \\
Measurement 2       & 0.762 \\
Measurement 3       & 0.773 \\
Measurement 4       & 0.784 \\
Measurement average & 0.768  \\ \hline
\textbf{Model measurement}  & \textbf{0.791} \\ \hline
\end{tabular} } \\
\caption[Short caption used in list of tables.]{\small{Longer caption to explain what the measurements are.}}
\end{table}

} % end font size change
\end{tcolorbox}
\end{singlespace}
\end{adjustwidth}

\end{document}

About

    About us
    Our values
    Careers
    Press & awards
    Blog

Learn

    LaTeX in 30 minutes
    Templates
    Webinars
    Tutorials
    How to insert images
    How to create tables

Plans & pricing

    Premium features
    For individuals & groups
    For enterprise
    For universities
    For students

Get involved

    Become an Overleaf advisor
    Participate in user research

Help

    Documentation
    Contact us
    Website status

© 2022 Overleaf
